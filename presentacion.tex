\documentclass[aspectratio=43]{beamer}

\usetheme{simple}

\usepackage{lmodern}
\usepackage[scale=2]{ccicons}

\usepackage[utf8]{inputenc}


\def\edicion{XXXV}
\def\fecha{Abril 2023 }

\title{Te has instalado Linux, ahora... ¿qué?} % Título de la presentación
\author{Luis Daniel Casais Mezquida} % autor de la presentación.
\github{rajayonin} % GitHub

\institute{\edicion \ Jornadas Técnicas del GUL}
\date{\fecha}


\titlegraphic{img/logo1.png}

\begin{document}

{
    \setbeamertemplate{footline}{}
    \begin{frame}
        \titlepage
    \end{frame}
}
\addtocounter{framenumber}{-1}


% -----------------------------
% La presentación empieza aquí.
% -----------------------------

\begin{frame}
    \frametitle{Tabla de contenidos}
    \tableofcontents
\end{frame}

% Contraseñas no se muestran en pantalla
% Desktop managers / tiling window managers (i3)
% LSW ("Linux Subsystem for Windows): Lutris, heroic games, steam, wine
% SSH
% Más gestores de paquetes: flatpak/nix, nala (pa debian) 
% Emuladores de terminal: kitty (bonito y simple, usa GPU), terminator (GUI friendly), alarcritty (moderno, usa GPU), iterm2 // tmux
% Shell scripts
% Funcionalidades de shell: pipes, &&, &, nohup, Ctrl-z, fg
% Comandos: find, who, top (htop) / ps, ping, traceroute, ipconfig, lscpu, grep, wget, echo, zip/unzip/tar, chmod, touch, cat, less, head, tail, sort, comm (diff/cmp), sed, history, sudo, df, service, kill/kilall, whereis, mount, fdisk/gparted
% Editores de texto en terminal: nano/pico, vim/nvim (lazyvim), emacs
% Reemplazos de comandos: exa, bat
% Ficheros de configuración: /etc/hosts
% Dotfiles: .bashrc, etc
% Configurar terminal: aliases
% Shells: zsh, fish
% Aiuda: man, --help, whatis


\begin{frame}
    \frametitle{}

\end{frame}


\end{document}
