\documentclass[aspectratio=43]{beamer}

\usetheme{simple}

\usepackage{lmodern}
\usepackage[scale=2]{ccicons}

\usepackage[utf8]{inputenc}


\def\edicion{XXXV}
\def\fecha{Abril 2023}

\title{Te has instalado Linux, ahora... ¿qué?} % título de la presentación
\author{Luis Daniel Casais Mezquida} % autor de la presentación
\github{rajayonin} % GitHub

\institute{\edicion \ Jornadas Técnicas del GUL}
\date{\fecha}


\titlegraphic{img/logo1.png}


% -----------------------------
% La presentación empieza aquí.
% -----------------------------


\begin{document}

{
    \setbeamertemplate{footline}{}
    \begin{frame}
        \titlepage
    \end{frame}
}
\addtocounter{framenumber}{-1}


\begin{frame}
    \frametitle{Tabla de contenidos}
    \tableofcontents
\end{frame}


\section{Personalización}
% Desktop Enviroments / tiling window managers (i3) / tty
% Emuladores de terminal: kitty (bonito y simple, usa GPU), terminator (GUI friendly), alarcritty (moderno, usa GPU), iterm2 // tmux
% Dotfiles: .bashrc, etc
% Shells: bash, zsh, fish

\subsection{Graphical User Interface}

\begin{frame}
    \frametitle{GUI (Graphical User Interface)}
    Por defecto, Linux cuenta con tres tipos principales de GUI: \textbf{Lockscreen} (pantalla de bloqueo), \textbf{Desktop Enviroment}, y \textbf{TTY/shell}.\newline

    Se puede acceder a cualquiera de ellos con las siguientes combinaciones de teclas:
    \begin{itemize}
        \item \texttt{CTRL + ALT + F1}: Lockscreen
        \item \texttt{CTRL + ALT + F2}: Desktop Environment
        \item \texttt{CTRL + ALT + F3}: TTY3
        \item \texttt{CTRL + ALT + F4}: TTY4\\
        ...\newline
    \end{itemize}
    
    Cualquiera de los tres tipos son personalizables e intercambiables.
\end{frame}

\subsection{Desktop Enviroment}
\begin{frame}
    \frametitle{Desktop Enviroment}

\end{frame}

\subsection{Tiling Window Managers}
\begin{frame}
    \frametitle{Tiling Window Managers}
    
\end{frame}

\section{Shell}
% Contraseñas no se muestran en pantalla, Ctrl-L, Ctrl-C, Ctrl-Shift-C, Ctrl-Shift-V
% Funcionalidades de shell: |, >>, >, &&, &, nohup, Ctrl-z, fg
% Shell scripts
% Comandos: find, who, top (htop) / ps, ping, traceroute, ipconfig, lscpu, grep, wget, echo, zip/unzip/tar, chmod, touch, cat, xxd, less, head, tail, sort, comm (diff/cmp), sed, history, sudo, df, service, kill/kilall, whereis, mount, fdisk/gparted (https://www.youtube.com/watch?v=s3ii48qYBxA, https://www.youtube.com/watch?v=PeCBpI1hT2Q&t=19s)
% Comandos coña: cowsays, fortune, sl, cmatrix, lolcat, fortune | cowsay -f tux | lolcat (https://www.youtube.com/watch?v=iAdpkqLD4f0)
% Editores de texto en terminal: nano/pico, vim/nvim (lazyvim), emacs
% Reemplazos de comandos: exa, bat
% Configurar terminal: aliases

\begin{frame}
    \frametitle{}
    
\end{frame}


\section{Más paquetes}
% Más gestores de paquetes: flatpak, snapd (malo), nix, nala (pa debian) 
% LSW ("Linux Subsystem for Windows"): wine, Lutris, heroic games, steam (proton)

\begin{frame}
    \frametitle{}
    
\end{frame}


\section{Configuración y ayuda}
% Ficheros de configuración: /etc/hosts
% Aiuda: man, --help, whatis

\begin{frame}
    \frametitle{}
    
\end{frame}


\section{Acceso remoto}
% ssh, scp


\begin{frame}
    \frametitle{}
    
\end{frame}


\end{document}
