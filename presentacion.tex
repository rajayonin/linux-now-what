\documentclass[aspectratio=43]{beamer}

\usetheme{simple}

\usepackage{lmodern}
\usepackage[scale=2]{ccicons}

\usepackage[utf8]{inputenc} % No olvides utilizar UTF8, especialmente si tu presentación está en castellano

\def\edicion{ % Cambiar para las siguientes jornadas
XXXV
}
\def\fecha{ % Cambiar para las siguientes jornadas
Abril 2023
}

\title{Te has instalado Linux, ahora... ¿qué?} % Modifica el título de la presentación a tu gusto
\author{Luis Daniel Casais Mezquida} % Cambia el autor de la presentación.
\github{rajayonin} % Pon tu GitHub o déjalo en blanco
\institute{\edicion Jornadas Técnicas del GUL} % Esto no lo cambies
\date{\fecha} % Esto tampoco

\titlegraphic{img/logo1.png}

\begin{document}

{
    \setbeamertemplate{footline}{}
    \begin{frame}
        \titlepage
    \end{frame}
}
\addtocounter{framenumber}{-1}

%\setwatermark{\includegraphics[height=8cm]{img/logo1.png}}

% ------------------------------------------------------------------------------------------
% La presentación empieza aquí. La primera diapositiva puede dejarse tal cual o sustituirse.
% ------------------------------------------------------------------------------------------

\begin{frame}
    \frametitle{Tabla de contenidos} % Puedes ponerle un título más chulo a la diapositiva con el índice de contenidos o dejar éste
    \tableofcontents % Para poblar la tabla de contenidos debes utilizar /section y /subsection apropiadamente (ver demo)
\end{frame}

% Contraseñas no se muestran en pantalla
% Shell script
% Desktop managers
% Ssh
% Lutris, heroic games, steam, wine, flatpak/nix, nala (pa debian) 
% Exa, bat
% pipes, &&, &, nohup, Ctrl-z, fg
% Bashrc / dotfiles en general
% Zsh, fish, kitty, terminator
% Man, --help


\begin{frame}
    \frametitle{}

\end{frame}


\end{document}
